\usepackage[dutch]{babel}
\usepackage{etex}
\usepackage[bookmarks=false]{hyperref}
\usepackage{easybmat}
\usepackage{fancyhdr}
\usepackage[explicit]{titlesec}
\usepackage{pgf,tikz}
\usetikzlibrary{shapes,arrows,positioning}
\usepackage{amsmath,amssymb,amsthm}
\usepackage{graphicx}
\usepackage{ifthen}
\usepackage{enumitem}
\usepackage{calc}
\usepackage{eurosym}
\usepackage{todonotes}
\usepackage{todo}
\usepackage{pdfpages}
\usepackage{color}
\usepackage{framed}
\usepackage{wrapfig}
\usepackage{listings}
\usepackage{changepage}
\usepackage{hyperref}
\usepackage{gensymb}
\usepackage{pdflscape}
\usepackage{smartdiagram}
\usesmartdiagramlibrary{additions}
\usepackage{xcolor}
\usepackage{pgfplots}
\usepackage{amsmath}

\pgfplotsset{compat=newest}
\usepackage{animate}
\usepackage{xsavebox} % xlrbox
\usepackage{calc} % \widthof{...}, \real{...}

\usepackage[breakable, theorems, skins]{tcolorbox}


\lstset{numbers=left, numberstyle=\tiny, stepnumber=1, numbersep=5pt, frame=tb}


\setlength{\oddsidemargin}{-15pt}
\setlength{\evensidemargin}{0pt}
\setlength{\topmargin}{-50pt}
\setlength{\textheight}{680pt}
\setlength{\textwidth}{485pt}
\setlength{\headsep}{30pt}
\setlength{\parindent}{0pt}
\setlength{\parskip}{1ex plus 0.5ex minus 0.2ex}
\setlength{\footskip}{40pt}
\setlength{\headheight}{50pt}
\addtolength{\baselineskip}{20pt}

\numberwithin{equation}{chapter}
\renewcommand{\theequation}{\arabic{chapter}.\arabic{equation}}

\newcommand{\bfit}{\bf\fontshape{it}\selectfont}

%%%%%%%%% De definities voor de headers %%%%%%%%%%%%%%%%%%%%%
%%%%%%%%% Juiste definitie voor input logo geven %%%%%%%%%%%%%%%%%%%%%%%%

%\newcommand{\filenaamlogo}{inputs/kubus-wetenschappen.png}
%\parbox[position][height][inner-pos]{width}{text}
%\newcommand{\insertlogo}{\parbox[c][0.95cm][c]{0.85cm}{\includegraphics[width=0.85cm]{\filenaamlogo}}}

\definecolor{fac-groen}{cmyk}{0.42,0,1,0}
%\definecolor{fac-groen-donker}{cmyk}{0.6,0.3,1,0}
\definecolor{fac-groen-donker}{cmyk}{0.4,0,1,0.38}
\definecolor{fac-color-orange}{cmyk}{0,0.38,0.99,0.13}

\newlength{\evenlengte}

\newcommand*\runningheader[1]{\settowidth{\evenlengte}{\mbox{\bf #1}}\addtolength{\evenlengte}{6mm}% 
{
\tikzset{external/export=false}
  \begin{tikzpicture}
    \fill[fill=fac-groen] (0,0) rectangle (\textwidth,1.3cm);
    \fill[fac-groen-donker] (\textwidth-\evenlengte,0) rectangle (\textwidth,1.3cm);
    %\node[xshift=1mm,anchor=south west]{\insertlogo};
    %\node[xshift=1.5cm,yshift=5.5mm,anchor=south west]{\color{white}\fontsize{11pt}{13pt}\selectfont\bf Junior College STEM};
    %\node[xshift=1.5cm,yshift=0.5mm,anchor=south west]{\color{white}\fontsize{11pt}{13pt}\selectfont\bf KU Leuven};
    \node[xshift=1mm,yshift=5.5mm,anchor=south west]{\color{white}\fontsize{11pt}{13pt}\selectfont\bf Junior College STEM};
    \node[xshift=1mm,yshift=0.5mm,anchor=south west]{\color{white}\fontsize{11pt}{13pt}\selectfont\bf KU Leuven};
    \node[xshift=\textwidth-2mm,yshift=0.5mm,anchor=south east]{{\color{white}\fontsize{11pt}{13pt}\selectfont\bf\boldmath #1}};
    \node[xshift=\textwidth-2mm,yshift=6mm,anchor=south east]{\color{white}\fontsize{11pt}{13pt}\selectfont\bf \thepage};
  \end{tikzpicture}}}

\newcommand*\hoofdstukheader{%
{\tikzset{external/export=false}
  \begin{tikzpicture}
    \fill[fill=fac-groen] (0,0) rectangle (\textwidth,1.3cm);
%     \node[xshift=1mm,anchor=south west]{\insertlogo};
%     \node[xshift=1.5cm,yshift=5.5mm,anchor=south west]{\color{white}\fontsize{11pt}{13pt}\selectfont\bf Junior College STEM};
%     \node[xshift=1.5cm,yshift=0.5mm,anchor=south west]{\color{white}\fontsize{11pt}{13pt}\selectfont\bf KU Leuven};
	\node[xshift=1mm,yshift=5.5mm,anchor=south west]{\color{white}\fontsize{11pt}{13pt}\selectfont\bf Junior College STEM};
    \node[xshift=1mm,yshift=0.5mm,anchor=south west]{\color{white}\fontsize{11pt}{13pt}\selectfont\bf KU Leuven};
    \node[xshift=\textwidth-2mm,yshift=6mm,anchor=south east]{\color{white}\fontsize{11pt}{13pt}\selectfont\bf \thepage};
  \end{tikzpicture}}}

\pagestyle{fancy}

\lfoot{}
\cfoot{}
\rfoot{}
\lhead{\runningheader{\nouppercase{\rightmark}}}
\chead{}
\rhead{}


\fancypagestyle{plain}{
  \fancyhf{}
  \lhead{\hoofdstukheader}
  \lfoot{}
  \cfoot{}
  \rfoot{}}



\renewcommand{\headrulewidth}{0pt}
\renewcommand{\footrulewidth}{0pt}

%%%%%%%%%% De definities voor de hoofdstukpagina's %%%%%%%%%%%%%%%%%%%%%%%%%%%%%%%%%

\newcommand*\chapterlabel{}

\newcommand{\chapterskip}{7mm}

\titlespacing*{\chapter}{0pt}{0pt}{0pt}

\titleformat{\chapter}[block]
  {\gdef\chapterlabel{} \huge\bfseries\boldmath}
  {\gdef\chapterlabel{\chaptertitlename}}{0pt}
  {\ifthenelse{\equal{\chapterlabel}{}}{\hoofdstukzondernaam{#1}}{\hoofdstukmetnaam{\chapterlabel}{#1}}
  \vspace{\chapterskip}}

\newcommand{\mijnchaptermark}{}
\renewcommand{\chaptermark}[1]{\markright{\mijnchaptermark}}

\newcommand{\hoofdstukzondernaam}[1]{%
\mbox{}\hspace{3.7cm}
\raisebox{20mm}[0cm]{
{\tikzset{external/export=false}
\begin{tikzpicture}[baseline=(current bounding box.north)]
 \node[anchor=north west,inner sep=4mm,very thick,draw=fac-groen-donker,text badly ragged,text width=12cm]{#1};
\end{tikzpicture}}%\vspace{1mm}
}}

\newcommand{\hoofdstukmetnaam}[2]{%
\mbox{}\hspace{3.7cm}
\raisebox{20mm}[0cm]{
{\tikzset{external/export=false}
\begin{tikzpicture}[baseline=(current bounding box.north)]
 \node[anchor=south west,rectangle,fill=fac-groen-donker]{\color{white}#1};
 \node[anchor=north west,inner sep=4mm,very thick,draw=fac-groen-donker,text badly ragged,text width=12cm]{#2};
\end{tikzpicture}}%\vspace{1mm}
}}


%%%%%%%%%%% De definities voor de stellingen, etc. %%%%%%%%%%%%%%%%%%%%%%%%%%%%%%%%%%%%%%

\newcommand{\stellingnaam}[1]{
{\tikzset{external/export=false}
\begin{tikzpicture}[baseline=(X.base)] \node[inner sep=0.6mm,rectangle,fill=fac-groen-donker](X){\color{white}#1}; \end{tikzpicture}}}

\newtheoremstyle{mijnstelling}% name
  {2ex}%      Space above, empty = `usual value'
  {2ex}%      Space below
  {\itshape}% Body font
  {}%         Indent amount (empty = no indent, \parindent = para indent)
  {\bfseries}% Thm head font
  {}%        Punctuation after thm head
  {2ex}%     Space after thm head: " " = normal interword space;
        %       \newline = linebreak
  {\ifthenelse{\equal{#2}{}}{\stellingnaam{#1}}{\stellingnaam{#1 #2}}}% Thm head spec

\newtheoremstyle{mijndefinitie}% name
  {2ex}%      Space above, empty = `usual value'
  {2ex}%      Space below
  {\rm}% Body font
  {}%         Indent amount (empty = no indent, \parindent = para indent)
  {\bfseries}% Thm head font
  {}%        Punctuation after thm head
  {2ex}%     Space after thm head: " " = normal interword space;
        %       \newline = linebreak
  {\ifthenelse{\equal{#2}{}}{\stellingnaam{#1}}{\stellingnaam{#1 #2}}}% Thm head spec

\newcommand{\voorbeeldnaam}[1]{
{\tikzset{external/export=false}
\begin{tikzpicture}[baseline=(X.base)] \node[inner sep=0.6mm,rectangle,fill=fac-color-orange](X){\color{white}#1}; \end{tikzpicture}}}

\newtheoremstyle{mijnvoorbeeld}% name
  {2ex}%      Space above, empty = `usual value'
  {2ex}%      Space below
  {\rm}% Body font
  {}%         Indent amount (empty = no indent, \parindent = para indent)
  {\it}% Thm head font
  {}%        Punctuation after thm head
  {2ex}%     Space after thm head: " " = normal interword space;
        %       \newline = linebreak
  {\ifthenelse{\equal{#2}{}}{\voorbeeldnaam{#1}}{\voorbeeldnaam{#1 #2}}}% Thm head spec	


% \newtheoremstyle{mijnvoorbeeld}% name
%   {2ex}%      Space above, empty = `usual value'
%   {2ex}%      Space below
%   {\rm}% Body font
%   {}%         Indent amount (empty = no indent, \parindent = para indent)
%   {\it}% Thm head font
%   {.}%        Punctuation after thm head
%   {2ex}%     Space after thm head: " " = normal interword space;
%         %       \newline = linebreak
%   {}% Thm head spec

{\theoremstyle{mijndefinitie}
\newtheorem*{definitie}{Definitie}}

{\theoremstyle{mijnstelling}
\newtheorem{eigenschap}{Eigenschap}[chapter]}
\renewcommand{\theeigenschap}{\arabic{eigenschap}}

{\theoremstyle{mijnvoorbeeld}
\newtheorem{voorbeeld}{Voorbeeld}[chapter]}
\renewcommand{\thevoorbeeld}{\arabic{voorbeeld}}

\newenvironment{bewijs eig}{\begin{proof}[Bewijs]}{\end{proof}}
\newenvironment{bewijs oplossing}{\begin{proof}[Uitgeschreven bewijs]}{\end{proof}}

%%%%%%%%%%%%% Abstracts van de hoofdstukken %%%%%%%%%%%%%%%%%%%%%%%%%%%%%%%

\newlength{\breedtesamenvatting}
\setlength{\breedtesamenvatting}{430pt}
\newlength{\samenvattingmarge}
\setlength{\samenvattingmarge}{(\textwidth - \breedtesamenvatting)/2}

\newenvironment{samenvatting}{\par\mbox{}\hspace{\samenvattingmarge}\begin{minipage}[t]{\breedtesamenvatting}}{\end{minipage}\vspace{3ex}\par}


%%%%%%%%%%%%% Er zijn geen hoofdstukken, maar wel module's, intermezzo's, etc, vandaar deze definities %%%%%%%%%%%%%%%%%%%%%%%%%%
%%%%%%%%%%%%% Het \deelmetoef veronderstelt dat de oplossingenfile's in gebruik zijn %%%%%%%%%%%%%%%%%%%%%%%%%%%%%%%%%%%%%%%%%%%%
%%%% De structuur is respectievelijk %%%
%%%%
%%%% \deelmetoef[optioneel argument met korte running header]{benaming van het deel met het nummer}{titel van het deel}{titel in table of contents}{titel in de oplossingenlijst}{running header in oplossingenfile} %%%
%%%%
%%%% \deelzonderoef[optioneel argument met korte running header]{benaming van het deel met het nummer}{titel van het deel}{titel in table of contents} %%%%
%%%%
%%%% \voorbereidendesessie[optioneel argument met korte running header]{benaming van de voorbereidende sessie}
%%%%
%%%% \oplossingenvoorbereidendesessie{Oplossingen van de ...}
%%%%

\renewcommand{\thechapter}{}

\newcommand{\mijnoplosmark}{}

\newcommand{\deelmetoef}[6][xyza]{\renewcommand{\chaptertitlename}{#2}%
\ifthenelse{\equal{#1}{xyza}}{\renewcommand{\mijnchaptermark}{#4}}{\renewcommand{\mijnchaptermark}{#1}}%
\chapter[#4]{#3}%
\immediate\write\alleoplossingen{\unexpanded{\renewcommand{\mijnoplosmark}}{#6}\unexpanded{\section}[#4]{#5}\unexpanded{\markright}{#6}}%
\immediate\write\oplossingenzonderster{\unexpanded{\renewcommand{\mijnoplosmark}}{#6}\unexpanded{\section}[#4]{#5}\unexpanded{\markright}{#6}}
}


\newcommand{\deelzonderoef}[4][xyza]{\renewcommand{\chaptertitlename}{#2}%
\ifthenelse{\equal{#1}{xyza}}{\renewcommand{\mijnchaptermark}{#4}}{\renewcommand{\mijnchaptermark}{#1}}%
\chapter[#4]{#3}}


\newcommand{\voorbereidendesessie}[2][xyza]{\renewcommand{\chaptertitlename}{Voorbereidende sessie}%
\ifthenelse{\equal{#1}{xyza}}{\renewcommand{\mijnchaptermark}{Voorbereidende sessie. #2}}{\renewcommand{\mijnchaptermark}{Voorbereidende sessie. #1}}%
\chapter[#1]{#2}
}

\newcommand{\oplossingenvoorbereidendesessie}[2]{\renewcommand{\chaptertitlename}{}\chapter{#1}\markright{#2}}

%%%%%%%%%%%% Figuren met een ondertitel %%%%%%%%%%%%%%%%%%%%%%%%%%%%%%%%%%%%%%%%%%%%%%%%%%%%%%%%%%%%%%%%%%%%%%%%%%%%%%%%%%%%%%%%%%%%%%%%%%%%%%%
%%%%%% De structuur is als volgt: \figuurmetlabel[\label{ref-figuur} - optioneel]{width=7cm}{filenaam.pdf}{Op deze figuur zie je...} %%%%%%%%%%
%%%%%% Gewone figuur: \gewonefiguur{width=7cm}{filenaam.pdf} %%%%%%%%%%%%%%%%%%%%%%%%%%%%%%%%%%%%%%%%%%%%%%%%%%%%%%%%%%%%%%%%%%%%%%%%%%%%%%%%%%

\newcounter{figuur}[chapter]
\renewcommand{\thefiguur}{\arabic{chapter}.\arabic{figuur}}
\newcommand{\figuurmetlabel}[4][]{\ifthenelse{\equal{#4}{}}%
{\begin{center}\includegraphics[#2]{inputs/#3}\nopagebreak\par\refstepcounter{figuur}Figuur \thefiguur#1\end{center}}%
{\begin{center}\includegraphics[#2]{inputs/#3}\nopagebreak\par\refstepcounter{figuur}Figuur \thefiguur.\ #4#1\end{center}}%
}


\newcommand{\gewonefiguur}[2]{\begin{center}\includegraphics[#1]{inputs/#2}\end{center}}

%%%%%%%%%%%% Tabellen %%%%%%%%%%%%%%%%%%%%%%%%%%%%%%%%%%%%%%%%%%%%%%%%%%%%%%%%%%%%%%%%%%%%%
%%%%%% De structuur is als volgt: \begin{tabel}{Al of niet de ondertitel van de tabel}\begin{tabular}[pos]{table spec} ... \end{tabular}\end{tabel} %%%%

\newcounter{tabelnr}[chapter]
\newcommand{\tabeltitel}{}
\renewcommand{\thetabelnr}{\arabic{chapter}.\arabic{tabelnr}}
\newenvironment{tabel}[1]{\begin{center}\refstepcounter{tabelnr}\renewcommand{\tabeltitel}{#1}}{\ifthenelse{\equal{\tabeltitel}{}}%
{\nopagebreak\vspace{1ex}\par Tabel \thetabelnr\end{center}}%
{\nopagebreak\vspace{1ex}\par Tabel \thetabelnr.\ \tabeltitel\end{center}}}

%%%%%%%%% Instelling van de opsommingen %%%%%%%%%%%%%%%

\setenumerate{leftmargin=5ex,labelsep=*,label=\arabic*.}

%%%%%%% Omgevingen waarin de oefeningen (oef en steroef) worden getypeset. %%%%%%%
%%%%%%% De omgevingen opgelost en opgelostster worden ingeroepen in de automatisch gegenereerde oplossingenfile om de oplossingen te typesetten. %%%%%%%%%%%

\newcommand{\kadertje}[1]{
\tikzset{external/export=false}
\begin{tikzpicture}[baseline=(X.base)] \node[inner sep=1mm,rectangle,fill=fac-groen-donker](X){\mbox{\color{white}\bf#1}}; \end{tikzpicture}}

\newlength{\oploslabel}
\setlength{\oploslabel}{8mm}
\newlength{\labelafstand}
\setlength{\labelafstand}{3mm}

\newenvironment{opgelost}[1]{\begin{list}{}{\setlength{\labelwidth}{\oploslabel}\setlength{\leftmargin}{\oploslabel}\addtolength{\leftmargin}{\labelafstand}\setlength{\labelsep}{\labelafstand}} \item[{\parbox[t]{\oploslabel}{\kadertje{#1}}}] }{\end{list}}

\newenvironment{opgelostster}[1]{\begin{list}{}{\setlength{\labelwidth}{\oploslabel}\setlength{\leftmargin}{\oploslabel}\addtolength{\leftmargin}{\labelafstand}\settowidth{\labelsep}{{\boldmath$\ast$}}\addtolength{\labelsep}{\labelafstand}} \item[{\parbox[t]{\oploslabel}{\mbox{}{\color{fac-groen-donker}{\boldmath$\ast$}}\kadertje{#1}}}] }{\end{list}}

\newcounter{oefen}[chapter]
\renewcommand{\theoefen}{\arabic{oefen}}

\newenvironment{somop}{\begin{list}{\parbox[t]{\oploslabel}{$\bullet$}}{\setlength{\labelwidth}{\oploslabel}\setlength{\leftmargin}{\oploslabel}\addtolength{\leftmargin}{\labelafstand}\setlength{\labelsep}{\labelafstand}}}{\end{list}}

\newenvironment{oef}{\begin{list}{}{\setlength{\labelwidth}{\oploslabel}\setlength{\labelsep}{\labelafstand}\setlength{\leftmargin}{\oploslabel}\addtolength{\leftmargin}{\labelafstand}}\refstepcounter{oefen} \item[{\parbox[t]{\oploslabel}{\kadertje{\theoefen}}}] }{\end{list}}

\newenvironment{steroef}{\begin{list}{}{\setlength{\labelwidth}{\oploslabel}\setlength{\leftmargin}{\oploslabel}\addtolength{\leftmargin}{\labelafstand}\settowidth{\labelsep}{{\boldmath$\ast$}}\addtolength{\labelsep}{\labelafstand}}\refstepcounter{oefen} \item[{\parbox[t]{\oploslabel}{\mbox{}{\color{fac-groen-donker}{\boldmath$\ast$}}\kadertje{\theoefen}}}] }{\end{list}}

%%%%%%%%%%% Sectietitels. Het commando \zondernummer wordt aangeroepen bij de opmaak van de %%%%%%%%%%%%%
%%%%%%%%%%% sectietitels bij de oplossingen. %%%%%%%%%%%%%%%%%%%%

\renewcommand{\thesection}{\arabic{section}}

\renewcommand{\sectionmark}[1]{\markright{#1}}

\newlength{\sectielabelbreedte}
\newlength{\sectielabelskip}
\newlength{\beschikbaar}
\newlength{\streepdikte}

\setlength{\sectielabelbreedte}{6mm}
\setlength{\sectielabelskip}{4mm}
\setlength{\streepdikte}{2mm}

\titleformat{\section}[block]{}{}{0pt}{\metnummer{\thesection}{#1}}

\newcommand{\metnummer}[2]{\setlength{\beschikbaar}{\textwidth-\sectielabelbreedte-\sectielabelskip-4mm}%
{\tikzset{external/export=false}
\begin{tikzpicture}
\fill[fill=fac-groen-donker] (0,0) rectangle (\textwidth,\streepdikte);
\node[anchor=north west,fill=fac-groen-donker,rectangle,inner sep=2mm]{\parbox[t]{\sectielabelbreedte}{\Large\bf\color{white}#1}};
\node[anchor=north east,xshift=\textwidth,yshift=-\streepdikte,inner sep=0pt]{\parbox[t]{\beschikbaar}{\Large\bf\boldmath#2}};
\end{tikzpicture}}}

\newlength{\streeplengte}
\setlength{\streeplengte}{9mm}

\newcommand{\zondernummer}[1]{\setlength{\beschikbaar}{\textwidth-2\streepdikte}%
\settoheight{\streeplengte}{\parbox[b]{\beschikbaar}{\Large\bf#1}}%
\addtolength{\streeplengte}{4mm}%
{\tikzset{external/export=false}
\begin{tikzpicture}
\fill[fill=fac-groen-donker] (0,0) rectangle (\textwidth,\streepdikte);
\fill[fill=fac-groen-donker] (0,0) rectangle (\streepdikte,-\streeplengte);
\node[anchor=north east,xshift=\textwidth,yshift=-\streepdikte,inner sep=0pt]{\parbox[t]{\beschikbaar}{\Large\bf\boldmath#1}};
\end{tikzpicture}}}


%%%%%%%% Formattering subsections %%%%%%%%%%%
\titleformat{\subsection}[block]{\large\bf\boldmath}{\thesubsection}{1.5ex}{#1}
%\titleformat{\subsection}[block]{}{}{0pt}{\large\bf\boldmath#1}
\titleformat{\subsubsection}[block]{}{}{0pt}{\normalsize\bf\boldmath#1}


%%%%%%%%%%%%%%%%%%%%%%%%%%%%%%%%%%%%%%%%%%%%%%%%%%
%%% Een specialleke bij Google: in te leveren opdracht %%%
%%%%%%%%%%%%%%%

\newenvironment{opdracht}[1]{\mijnopdracht{#1}}{\vspace{3ex}\par}

\newcommand{\mijnopdracht}[1]{{\titleformat{\subsection}[block]{}{}{0pt}{\opdrachttypeset{#1}}%
\subsection*{#1}}}

\newlength{\overschot}
\newlength{\linkerlijn}
\newlength{\eenbreedte}
\setlength{\linkerlijn}{1cm}

\newcommand{\opdrachttypeset}[1]{\settowidth{\eenbreedte}{\bf\large\boldmath#1}%
\setlength{\overschot}{\textwidth-\linkerlijn-\eenbreedte-4ex}%
{\tikzset{external/export=false}
\begin{tikzpicture}
\fill[fill=fac-groen-donker] (0,\streepdikte/2) rectangle (\linkerlijn,-\streepdikte/2);
\node at (\linkerlijn+2ex,0) [anchor=west,inner sep=0pt] {{\bf\large\boldmath#1}};
\fill[fill=fac-groen-donker] (\textwidth-\overschot,\streepdikte/2) rectangle (\textwidth,-\streepdikte/2);
\end{tikzpicture}}}



%% Commands door Liesje
\usepackage[bottom]{footmisc}
\newcommand{\gewonefiguurzc}[2]{\newline \includegraphics[#1]{inputs/#2} }%\newline}
\newcommand{\gewonefiguurnocenter}[2]{\includegraphics[#1]{inputs/#2}}

\newcommand*\circled[1]{\tikz[baseline=(char.base)]{
            \node[shape=circle,draw,inner sep=1.5pt] (char) {#1};}}
           
\newcommand{\R}{\mathbb{R}}

\newtheoremstyle{mijnopmerking}% name
  {2ex}%      Space above, empty = `usual value'
  {2ex}%      Space below
  {\rm}% Body font
  {}%         Indent amount (empty = no indent, \parindent = para indent)
  {\bfseries}% Thm head font
  {}%        Punctuation after thm head
  {2ex}%     Space after thm head: " " = normal interword space;
        %       \newline = linebreak
  {\ifthenelse{\equal{#2}{}}{\stellingnaam{#1}}{\stellingnaam{#1 #2}}}% Thm head spec


\lstset{breaklines=true}

{\theoremstyle{mijnopmerking}
\newtheorem*{opmerking}{Opmerking}}
      

%%%%%%%%%%%%%%%%%%%%%%%%%%%%%%%%%%%%%%%%%%%%%%%%%%%%%
